\chapter{Introduction}

\section{Background}

Crisis Mapping is the process of collecting, analyzing and visualizing crisis data for further crisis management. The goal is to provide better real time decision support to determine the best course of action in a changing or evolving crisis environment. This in turn provide help to crisis victims and save more lives \cite{irevol}.
\\\\
In 2010, a major earthquake crisis struck Haiti. The shock measured 7.0 on the moment magnitude scale (MMS) and was followed by another 52 aftershocks of 4.0 MMS or higher over 12 days. 1 517 000 people and their dependents were affected. People were in need of housing, food and medical attention. Mapping these needs would improve further crisis management \cite{usha}.
\\\\
The needs on Haiti were mapped by Ushahidi\cite{usha}, an open source software. This software collects, analyzes and visualizes data from social media, text messages and more. The Haiti government took advantage of this software and allowed crisis victims and bystanders to report needs to temporary web services and text message services. These services together with Ushahidi provided a crisis map that aided the government to send help where it was needed \cite{usha}.
\\\\
Today, researchers are investigating how to use mobile phone sensors to map crisis situations. Many mobile phones have built-in sensors such as microphones, accelerometers, gyroscopes, barometers and GPS. In theory, these sensors can be used to detect earthquakes, fires, explosions, collisions and other crisis situations by reading abnormal magnitudes or patterns in sound, acceleration, air pressure and temperature. Mobile phones can be used to report this abnormality to a central crisis mapping system and possibly aid further crisis management.
\\\\
\section{Problem Statement}

CISTECH of the University of Agder proposed a basic project about crisis mapping using mobile phones. CISTECH is  Centre for Information Systems Technology in Emergency Preparedness \& Management. Because crisis mapping is a general term and includes a broad range of different crisis situations, we had to narrow down the goal of the project. After consulting with our supervisor, the final goal was to create a prototype crisis mapping system that could detect fire using mobile phone sensors. Fire was chosen specifically because it can change or evolve over time. However, the chosen topic does not come without a set of practical limitations.

Given the project goal, we focused on two main problems that had to be solved. Firstly, fires had to be simulated. There was no way to justify starting real fires to gather this data. Simulating fire is challenging because fire spreads based on several factors such as wind, heat, humidity and others over time. Secondly, mobile phone sensors do only report readings at their location and do not cover burning areas without nearby sensors. Uncovered areas have to be guesstimated.

It was necessary to set a few project limitations to make sure the project kept on track. The simulation will not attempt to use real sensor data from a phone, but will use an abstract approach to sensing. The phones senses fires based on the intensity or they does not detect any fire. Most smart phones are not equipped with fire and smoke detectors, but if a smart phone were to have such a feature it would most likely drain the battery rapidly. In this project a sensor is meant to be a person with a phone which can detect fire. This person never gets tired and his or her smart phone never runs out of battery.

\section{Literature Review}

\subsection{A Survey of Mobile Phone Sensing}

The field of mobile phone sensing is rapidly expanding \cite{mobsurv}. The accelerometer in a smart phone can determine whether the user is walking, running or standing still. The microphone can collect audio. The light sensors can determine whether the phone is in a pocket or not and the barometer can determine air pressure. In addition data can be combined from different sensors to make assumptions, for instance a phone can sense that a user is walking down a hill and in what direction. Specialized sensors have also been used to measure air pollution and could perhaps be used to detect explosive residue.

There are many issues with properly utilizing and gathering data. First of all a large amount of users to properly test a system is needed. This can be solved by distributing the application via the App stores and providing an incentive for the users to use it. Second there are huge privacy issues with listening in on private conversations using the microphone and tracking the user via the GPS. This can somewhat be fixed by calculating any result on the phone and sending only the results, however this would require more resources from the phone and would likely limit battery time or performance. Some studies have shown that sensing systems have reduced the standby time by significant amounts, vastly affecting the user experience.

Creating a system that produce accurate results may also be slightly problematic as vendors did not foresee the use of their sensors in new applications. The sensor sampling rate varies based on the CPU load and the API and OS does not support complete access to the sensors. However, as the mobile phone sensing field is growing vendors are likely to catch on and provide better sensors and sensor support in future releases.

Moreover, it is difficult to create a program that fully accounts for all possible instances. Explosives used in demolition and terrorism both create the same or similar sound and gun fire while hunting sounds the same as gun fire in a super marked. A phone also needs to know if it is currently inside a container where the sound might be suppressed, in which case the system might become more sensitive to sound. This in turn can be somewhat solved by using light or proximity sensors however there might be cases where the phone is placed in an unusual location and therefore provides inaccurate results.

\subsection{Distributed Perception Networks for Crisis Management}

Distributed Perception Networks (DPN) for Crisis management (2006)\cite{dpn} focuses on the fusion of information gathered from fixed sensors and human perception. The fixed sensors are few and scarcely distributed in an area where a crisis could have devastating consequences. Endsley’s model tells that first comes perception, then comprehension and lastly projection. The paper focuses on combining the perception of humans and the fixed sensors to map the crisis situation. For example when the sensors are picking up dangerous levels of ammonia in the air, the system will use human perception to gather more input. This is done by sending SMS to everyone in that area where the query could be; “Do you smell ammonia?”. By using human perception the crisis management team will get a richer and more detailed map instead of only using the scarcely distributed fixed sensors.

Sensor readings and public perception results in a vast amount of information which needs to be processed. Since it is such an enormous task for manual human labour the paper proposes to use a Bayesian network to find the most probable cause of the crisis. There is a large degree of uncertainty when gathering huge amounts of information from people and possibly faulty sensors. The DPN architecture represents a multi agent system which means it can comprehend large problems in an efficient manner. A set of nodes is processed by one agent before they send it to their father node. The paper does not suggest how to calculate how toxic the air will become or where it will spread.  

\subsection{Video about how Crisis Mapping Helped During the Haiti Earthquake}

When the earthquake hit Haiti in 2010, there was an initiative to map where help was needed \cite{usha}. This increased the efficiency of the rescue personnel. In addition, no emergency call centers were operational directly after the earthquake. Resulting in the crisis map to become an invaluable tool for organizing the aid. People in need of help would send a text message signifying their location and situation, operators would then note what help was needed and place the note on the map showing the location. Organisations could then use this map to deploy personnel in areas where they were needed the most. In addition, any volunteers could view the map and help if they were able.

One thing to note about this system was that messages were written in different languages and then translated to English. Most of the translation were done with assistance from the general public, this highlighted the positives of a crisis map which is open to the public.

By using a map and plot what is happening in the different areas, organisations that are trying to help may send their resources to the correct locations at the right time. Organisations may also inform people where dangerous locations are during the crisis and guide the people in that area to a safe location. In addition, they may plan their next move as they observe the evolving crisis. People that are able to see this crisis map may see the locations that are dangerous, for example they can then use the map to spot a burning building and head in a different direction. 

\subsection{A Real-time Disaster Situation Mapping System for University Campuses}

A real-time disaster situation mapping (RDSM) system is a web-based system which acts as a social media. The system helps its users and the disaster management team during a crisis situation by giving the disaster management team a clear overview of the situation and the users can use the system to acquire their location and the nearest exit.

The RDSM system is based on some conditions such as disaster type, place of use and targeted users. The main focus of the system is to protect people during and immediately after a disaster based on the assumptions that the earthquake has a minimal impact on the university buildings. In the event that the university buildings are severely damaged then important equipment such as servers, Wireless LAN, etc. may be out of commission and the system may end up not working properly. The mobile devices, such as computers or phones, are assumed to be in use and connected to the servers during the disaster situation. The RDSM system can be accessed inside the university as well as in the evacuation office by any employee with sufficiently high authority level.

The RDSM system has two major parts, called situation gathering sub system (SGS) and the situation mapping sub system (SMS) \cite{springlink}. The SGS allows people in the campus to send situation information about their location, such as degree of longitude, latitude and name of the place at the time of the disaster, using their mobile devices.The location information can be identified by manual entry as well as by the automatic acquisition using Place Engine. Place Engine is a service where users can send location information, even though they do not recognize their locations, by utilizing Wi-Fi devices. Some requirements have to be satisfied in order to set up the Place Engine for automatic data acquisition of location information. First, the Wireless LAN access point data must be saved in the Place Engine database. Secondly, at least one access point must be reachable from the users  location. Next, access to the Place Engine server must be feasible. Finally, the users’ mobile device(s) (PC or phone) must be equipped with Wireless LAN (Wi-Fi) equipment and the client software must be installed on those mobile device(s) \cite{springlink}. The users can send situation information from the situation data entry screen by accessing the SGS. Speech to text transformation is achieved by using “w3voice”, which is a development kit for voice-enabled web applications \cite{springlink}. Therefore the informers can send voice data, text data as well as video via the disaster situation entry screen.The entered situation data is stored in the situation database. 

The SMS reads data from the situation data base updated by SGS and it will replicate that data onto maps sent by the users. The more detailed information of the disaster situation is shown by the time series on the map screen. The users can also access the voice or image data  in order to get a clear and accurate information about the crisis situation.Situation data that are described by text data are rendered on this “png” map in SMS. In order to show the disaster affected areas (points) on the map, Google maps are used to get the degrees of latitude and longitude of two different points. “By calculating the distance between user’s location and the point obtained from Google maps, users’ location is identified from the reduced scale distance that is started from two different points” \cite{springlink}.

\section{Solution Approach}

In this project, we simulated our crisis mapping system using two grids. Both grids represented the same area (any area) divided into equally sized squared cells. In the first grid, we let fire and mobile phone sensors spread freely over the cells. In the second grid, we estimated the fire situation based on mobile sensor readings from the first grid using Gaussian Processes Regression (GPR). Having the actual situation put next to the estimated situation, we could easily verify the strength and weaknesses of both the crisis mapping system and concept in terms of similarity and differences.

\section{Report Outline}

Chapter one introduces crisis mapping, what problem this project is trying to solve and the solution approach. In addition the chapter contains literature reviews of relevant papers. The second chapter of the report contains the theoretical background which the project is based on. The fire theory is used in the implementation of the fire simulation and the theory on Gaussian Processes is used in the implementation of the interpreter.

Chapter tree describes the project requirements and limitations. Additionally it outlines what the simulation and interpreter were expected to do, how the expected functions were implemented and if our way of implementing the features were the correct one. The fourth chapter contains the discussion which aims to present the results, explain why the solution approach was chosen over other alternative solutions and how well the solution satisfied the requirements.

Finally, chapter five contains the conclusion. It endeavours to determine whether the results makes a difference to existing solutions and what the benefits are for system users. Additionally it contains proposed next steps if the project were to be continued.