\chapter{Conclusion}

The goal of this project was to help the users determine the best course of action during an evolving crisis. As crisis mapping includes a fairly broad range of crisis situations the goal was to create a prototype crisis mapping system that could detect and map fire using mobile phone sensors. This system were to exclusively use simulated data as setting up mobile phone sensors to properly detect a fire was outside of the project scope. To reach the project goals the system would utilize two grids side by side, one containing the simulated crisis and the other would visualize the data predicted by the Gaussian Processes.

In the end the project met the requirements. In a simple environment the fire interpreter produces good results while in the advanced simulation the fire interpreter produces an accurate shape, but the size of the estimated fire is a little larger. When more environmental factors are introduced to the system the results are less accurate, however the overall performance is quite good.

While the system itself is not ready for deployment it provides insight into the problem and a starting point for future work. The project was never meant to become a finished product. The system is a  proof-of-concept that demonstrates the validity of using Gaussian Processes to predict fire spread. The system provides an easy way of estimating results and as such provides a good framework for future work. There are a few improvements that could be made to advance the current solution.

Firstly the simulation would need to implement a large variety of different squares. In the real world fire is hard to predict as it is affected by several factors. To properly simulate a real fire the simulation would need to simulate a real geographical area with houses, ocean, rivers and so on. Secondly fire would need to behave according to the current understanding of fire and fire movement. I.e., fire moves faster uphill than downhill, there are different types and shapes of fire etc.

Thirdly the simulated sensors move randomly in the current version of the system. To improve upon this it could be beneficial to look into human movement during a crisis and have the sensors move accordingly. Finally, if possible the system should receive and interpret real data from real sensors. This would naturally be the best solution, however the system would likely need to go through several versions of the simulator before it could be used to interpret vast amounts of real data.