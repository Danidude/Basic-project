
\documentclass[pdftex,12pt,final,a4paper]{report}

% This is for norwegian letters (æøå)
\usepackage[utf8]{inputenc}

% This is to use \includegraphics to include the domain model
\usepackage[pdftex]{graphicx}

%\usepackage{setspace}
%\onehalfspace

% Use to create acronym list
\usepackage{acronym}
% A package that supports a lot of math symbols
\usepackage{amssymb,amsmath}
% Times font
\usepackage{times}
% This enables multirows in tables
\usepackage{multirow}
%enable float on placement on figures and tables
\usepackage{float}
%This package enables active links
\usepackage[pdftex]{hyperref}
\hypersetup{
    colorlinks,%
    citecolor=black,%
    filecolor=black,%
    linkcolor=black,%
    urlcolor=black
}

% For pseudocode and algorithms
%\usepackage{algorithmicx}
%\usepackage{algorithmic}
%\usepackage{algorithm}

%\usepackage{algorithm}
%\usepackage{algpseudocode}


% This adds line breaks between paragraphs
\setlength{\parskip}{0.10in}
% This enables fancy header
\headheight=14.5 pt
\usepackage{fancyhdr}
\pagestyle{fancy}
\fancyhead[L]{\leftmark}
\fancyhead[R]{}

% Enables UiA frontpage
\usepackage{uiafrontpage} 

%Enables subfigures
%\usepackage{subfig}


%%%%%%%oursss
\usepackage[USenglish]{babel}
%\usepackage{qtree}
\usepackage{verbatim}

%\usepackage[toc,page]{appendix}


\author{Mari Næss \\
Ørjan Hatteberg \\
Enok Karlsen Eskeland}

%\supervisor{Folke Haugland \\ Jaran Nilsen}

% select logo to use
\titlelogo{introduction/graphics/uia_logo.png}





\begin{document}

\title{Distributed querying with Apache Solr}
\maketitle

\vspace{2in}
\begin{abstract}
Crisis mapping is a concept to help crisis management teams to determine the best course of action in a changing or evolving crisis situation in order to help people and save more lives. This report describes how mobile phone sensors can be used to measure, interpret and visualize a fire crisis situation. An interactive simulator was especially designed for this project to provide sensor readings affected by environmental parameters such as wind and humidity. These individual readings were used in the Gaussian Processes Regression (GPR) to estimate the area and magnitude of the fire. The kernel function of the GPR was expanded to also take wind into account to the distance vectors.. The overall system performance were visualized by putting the actual situation from the simulator together with the estimated situation from the GPR. The final results of this work shows acceptable accuracy and serve as a proof-of-concept that mobile phones can be used to map fire crisis situation. 
\end{abstract} 





%\include{preface}
%\include{thesisDefinition}

% Create table of contents 
\tableofcontents 

\listoffigures

\listoftables

%\listofalgorithms

\pagebreak
\thispagestyle{plain}
\begin{Huge}
\label{definitions}
\begin{flushleft}

\bf Definition list
\end{flushleft}
\end{Huge}

\begin{description}

  \item[Taxonomy] is by Integrasco usage and in this context defined as a complex query.
  \item[Document] is the basic unit in Lucene indexing. E.g. a single pdf or a book.
  \item[Rows] is the number of documents in the result set of a query.
  \item[Start Offset] is the index of the first document you want displayed.
  \item[Page Offset] is used in pagination, but is the same as start offset.
  \item[Iterations] are the number of times a taxonomy is queried.
  \item[Hit Count] is the total number of documents matching the query.
  \item[QueryOptimizer] library is the solution developed for the problem.
  \item[QTime] is the time spent generating the in memory response for a query in Solr (milliseconds).
  \item[Elapsed Time] is QTime plus serializing and de-serializing transmitting in Solr (milliseconds).
  \item[Query Time] is the time it takes to perform a solr search from QueryOptimizer or the test framework (milliseconds).
  \item[Lucene] is an open source free text search library from Apache.
  \item[Solr] is an open source search server utilizing Lucene.
  \item[Solrconfig.xml] contains the parameters to configure Solr.
  \item[QueryResultWindowSize]. A window is a section of search results. It can be from 0-49, 50-99 etc. When querying the entire window in which the search match will be returned and loaded into cache. QueryResultWindowSize is the size of these windows.
  \item[QueryResultMaxDocsCached] is the maximum number of documents a single query can have in cache memory.
  \item[Index] is a sorted list of terms present in the data set. Contains links for finding the term locations.
  \item[Sharded index] is an index split in smaller parts possibly on different servers to better cope with scaling issues.
\end{description}

%\chapter{Introduction}

\section{Background}

Crisis Mapping is the process of collecting, analyzing and visualizing crisis data for further crisis management. The goal is to provide better real time decision support to determine the best course of action in a changing or evolving crisis environment. This in turn provide help to crisis victims and save more lives \cite{irevol}.
\\\\
In 2010, a major earthquake crisis struck Haiti. The shock measured 7.0 on the moment magnitude scale (MMS) and was followed by another 52 aftershocks of 4.0 MMS or higher over 12 days. 1 517 000 people and their dependents were affected. People were in need of housing, food and medical attention. Mapping these needs would improve further crisis management \cite{usha}.
\\\\
The needs on Haiti were mapped by Ushahidi\cite{usha}, an open source software. This software collects, analyzes and visualizes data from social media, text messages and more. The Haiti government took advantage of this software and allowed crisis victims and bystanders to report needs to temporary web services and text message services. These services together with Ushahidi provided a crisis map that aided the government to send help where it was needed \cite{usha}.
\\\\
Today, researchers are investigating how to use mobile phone sensors to map crisis situations. Many mobile phones have built-in sensors such as microphones, accelerometers, gyroscopes, barometers and GPS. In theory, these sensors can be used to detect earthquakes, fires, explosions, collisions and other crisis situations by reading abnormal magnitudes or patterns in sound, acceleration, air pressure and temperature. Mobile phones can be used to report this abnormality to a central crisis mapping system and possibly aid further crisis management.
\\\\
\section{Problem Statement}

CISTECH of the University of Agder proposed a basic project about crisis mapping using mobile phones. CISTECH is  Centre for Information Systems Technology in Emergency Preparedness \& Management. Because crisis mapping is a general term and includes a broad range of different crisis situations, we had to narrow down the goal of the project. After consulting with our supervisor, the final goal was to create a prototype crisis mapping system that could detect fire using mobile phone sensors. Fire was chosen specifically because it can change or evolve over time. However, the chosen topic does not come without a set of practical limitations.

Given the project goal, we focused on two main problems that had to be solved. Firstly, fires had to be simulated. There was no way to justify starting real fires to gather this data. Simulating fire is challenging because fire spreads based on several factors such as wind, heat, humidity and other over time. Secondly, mobile phone sensors do only report readings at their location and do not cover burning areas without nearby sensors. Uncovered areas have to be guesstimated.

\section{Literature Review}

\subsection{A Survey of Mobile Phone Sensing}

The field of mobile phone sensing is rapidly expanding \cite{mobsurv}. The accelerometer in a smart phone can determine whether the user is walking, running or standing still. The microphone can collect audio. The light sensors can determine whether the phone is in a pocket or not and the barometer can determine air pressure. In addition data can be combined from different sensors to make assumptions, for instance a phone can sense that a user is walking down a hill and in what direction. Specialized sensors have also been used to measure air pollution and could perhaps be used to detect explosive residue.

There are many issues with properly utilizing and gathering data. First of all a large amount of users to properly test a system is needed. This can be solved by distributing the application via the App stores and providing an incentive for the users to use it. Second there are huge privacy issues with listening in on private conversations using the microphone and tracking the user via the GPS. This can somewhat be fixed by calculating any result on the phone and sending only the results, however this would require more resources from the phone and would likely limit battery time or performance. Some studies have shown that sensing systems have reduced the standby time by significant amounts, vastly affecting the user experience.

Creating a system that produce accurate results may also be slightly problematic as vendors did not foresee the use of their sensors in new applications. The sensor sampling rate varies based on the CPU load and the API and OS does not support complete access to the sensors. However, as the mobile phone sensing field is growing vendors are likely to catch on and provide better sensors and sensor support in future releases.

Moreover, it is difficult to create a program that fully accounts for all possible instances. Explosives used in demolition and terrorism both create the same or similar sound and gun fire while hunting sounds the same as gun fire in a super marked. A phone also needs to know if it is currently inside a container where the sound might be suppressed, in which case the system might become more sensitive to sound. This in turn can be somewhat solved by using light or proximity sensors however there might be cases where the phone is placed in an unusual location and therefore provides inaccurate results.

\subsection{Distributed Perception Networks for Crisis Management}

Distributed Perception Networks (DPN) for Crisis management (2006)\cite{dpn} focuses on the fusion of information gathered from fixed sensors and human perception. The fixed sensors are few and scarcely distributed in an area where a crisis could have devastating consequences. Endsley’s model tells that first comes perception, then comprehension and lastly projection. The paper focuses on combining the perception of humans and the fixed sensors to map the crisis situation. For example when the sensors are picking up dangerous levels of ammonia in the air, the system will use human perception to gather more input. This is done by sending SMS to everyone in that area where the query could be; “Do you smell ammonia?”. By using human perception the crisis management team will get a richer and more detailed map instead of only using the scarcely distributed fixed sensors.

Sensor readings and public perception results in a vast amount of information which needs to be processed. Since it is such an enormous task for manual human labour the paper proposes to use a Bayesian network to find the most probable cause of the crisis. There is a large degree of uncertainty when gathering huge amounts of information from people and possibly faulty sensors. The DPN architecture represents a multi agent system which means it can comprehend large problems in an efficient manner. A set of nodes is processed by one agent before they send it to their father node. The paper does not suggest how to calculate how toxic the air will become or where it will spread.  

\subsection{Video about how Crisis Mapping Helped During the Haiti Earthquake}

When the earthquake hit Haiti in 2010, there was an initiative to map where help was needed \cite{usha}. This increased the efficiency of the rescue personnel. In addition, no emergency call centers were operational directly after the earthquake. Resulting in the crisis map to become an invaluable tool for organizing the aid. People in need of help would send a text message signifying their location and situation, operators would then note what help was needed and place the note on the map showing the location. Organisations could then use this map to deploy personnel in areas where they were needed the most. In addition, any volunteers could view the map and help if they were able.

One thing to note about this system was that messages were written in different languages and then translated to English. Most of the translation were done with assistance from the general public, this highlighted the positives of a crisis map which is open to the public.

By using a map and plot what is happening in the different areas, organisations that are trying to help may send their resources to the correct locations at the right time. Organisations may also inform people where dangerous locations are during the crisis and guide the people in that area to a safe location. In addition, they may plan their next move as they observe the evolving crisis. People that are able to see this crisis map may see the locations that are dangerous, for example they can then use the map to spot a burning building and head in a different direction. 

\subsection{A Real-time Disaster Situation Mapping System for University Campuses}

A real-time disaster situation mapping (RDSM) system is a web-based system which acts as a social media. The system helps its users and the disaster management team during a crisis situation by giving the disaster management team a clear overview of the situation and the users can use the system to acquire their location and the nearest exit.

The RDSM system is based on some conditions such as disaster type, place of use and targeted users. The main focus of the system is to protect people during and immediately after a disaster based on the assumptions that the earthquake has a minimal impact on the university buildings. In the event that the university buildings are severely damaged then important equipment such as servers, Wireless LAN, etc. may be out of commission and the system may end up not working properly. The mobile devices, such as computers or phones, are assumed to be in use and connected to the servers during the disaster situation. The RDSM system can be accessed inside the university as well as in the evacuation office by any employee with sufficiently high authority level.

The RDSM system has two major parts, called situation gathering sub system (SGS) and the situation mapping sub system (SMS) \cite{springlink}. The SGS allows people in the campus to send situation information about their location, such as degree of longitude, latitude and name of the place at the time of the disaster, using their mobile devices.The location information can be identified by manual entry as well as by the automatic acquisition using Place Engine. Place Engine is a service where users can send location information, even though they do not recognize their locations, by utilizing Wi-Fi devices. Some requirements have to be satisfied in order to set up the Place Engine for automatic data acquisition of location information. First, the Wireless LAN access point data must be saved in the Place Engine database. Secondly, at least one access point must be reachable from the users  location. Next, access to the Place Engine server must be feasible. Finally, the users’ mobile device(s) (PC or phone) must be equipped with Wireless LAN (Wi-Fi) equipment and the client software must be installed on those mobile device(s) \cite{springlink}. The users can send situation information from the situation data entry screen by accessing the SGS. Speech to text transformation is achieved by using “w3voice”, which is a development kit for voice-enabled web applications \cite{springlink}. Therefore the informers can send voice data, text data as well as video via the disaster situation entry screen.The entered situation data is stored in the situation database. 

The SMS reads data from the situation data base updated by SGS and it will replicate that data onto maps sent by the users. The more detailed information of the disaster situation is shown by the time series on the map screen. The users can also access the voice or image data  in order to get a clear and accurate information about the crisis situation.Situation data that are described by text data are rendered on this “png” map in SMS. In order to show the disaster affected areas (points) on the map, Google maps are used to get the degrees of latitude and longitude of two different points. “By calculating the distance between user’s location and the point obtained from Google maps, users’ location is identified from the reduced scale distance that is started from two different points” \cite{springlink}.

\section{Solution Approach}

In this project, we simulated our crisis mapping system using two grids. Both grids represented the same area (any area) divided into equally sized squared cells. In the first grid, we let fire and mobile phone sensors spread freely over the cells. In the second grid, we estimated the fire situation based on mobile sensor readings from the first grid using Gaussian Processes Regression (GPR). Having the actual situation put next to the estimated situation, we could easily verify the strength and weaknesses of both the crisis mapping system and concept in terms of similarity and differences.

\section{Report Outline}

Chapter one introduces crisis mapping, what problem this project is trying to solve and the solution approach. In addition the chapter contains literature reviews of relevant papers. The second chapter of the report contains the theoretical background which the project is based on. The fire theory is used in the implementation of the fire simulation and the theory on Gaussian Processes is used in the implementation of the interpreter.

Chapter tree describes the project requirements and limitations. Additionally it outlines what the simulation and interpreter were expected to do, how the expected functions were implemented and if our way of implementing the features were the correct one. The fourth chapter contains the discussion which aims to present the results, explain why the solution approach was chosen over other alternative solutions and how well the solution satisfied the requirements.

Finally, chapter five contains the conclusion. It endeavours to determine whether the results makes a difference to existing solutions and what the benefits are for system users. Additionally it contains proposed next steps if the project were to be continued.
%\include{./background/background}
%\include{./testing/tests}
%\chapter{Solution}
\section{Fire and Sensor Simulation}
\section{Fire Interpreter}

The fire interpreter's job is to receive input from the simulator and 
calculate values for the received cells. The calculated values are then posted to the visualizer. The advanced calculations are done by the library pyXGPR. This is a Gaussian Process Regression library implemented with Python. It produces a mean and a variance when used correctly. The first input parameter \textbf{X} is a list of points which tells where the training data is located. Another parameter \textbf{Y} contains the values to the training data. The last interpreter generated parameter \textbf{x star} contains the points where we want to find the mean and the variance. In addition to these parameters the library needs to be told what covariance functions pyXGPR should use to calculate the correlation between the cells in \textbf{X}, \textbf{Y} and \textbf{x star}. There is also added parameter values to these functions.
\\\\
The most basic use of pyXGPR is one dimensional (line regression) where \textbf{X} is the location and \textbf{Y} is the value. The fire interpreter uses regression in three dimensions where \textbf{X} and \textbf{Y} are the map coordinates and an additional parameter \textbf{t} is for time. \textbf{t} is necessary to save earlier sensor data which later are utilized in calculations. It should also be mentioned that before this implementation, this was done by saving the best data. Best data is to be understood as the data which has the lowest variance. Data with lower variance would be applied to the saved map. This hack and the implementation of t is done because previous sensor data is important as long as they are weighted less than the newest sensor data. As time increases there will be sensor data covering most of the map, but the old sensor data will have less weight and thus giving new sensor data the opportunity to be taken into account. Figure \ref{fig:timeElapse} illustrates this.


\begin{figure}[here]
  \centering
      \includegraphics[width=0.5\textwidth]{solution/graphics/timeElapse.png}
  \caption{NEED TO SET A USEFUL CAPTION. NEED TO SET A USEFUL CAPTION}
  \label{fig:timeElapse}
\end{figure}


The simulation map is 71 X 71 squares. GPR calculates the correlation between all points. This means pyXGPR creates huge matrices. As t increases the the matrices grow larger and thus the execution time rises. The first version of the program used more than 6 minutes to run on a normal laptop. To better the performance it was found necessary to locate some squares which did not need to be calculated. Cells containing sensors was therefore removed from \textbf{x star} since the values were already there. With the introduction of t the matrices became larger and execution time rose and once again it was necessary to find some steps to improve the performance. For a cell to be added to \textbf{x star} it had to be close to sensor which communicated fire. This was done by finding the euclidean distance between all cells which had no value and all sensors which communicated fire. This reduced the number of points in \textbf{x star} significantly. As time increases Another measure taken to reduce the execution time was to save cells which was calculated to be on fire. From these the euclidean distance to all points were calculated and if they were within a certain distance they would be added to \textbf{x star}.
\\\\
The simulation use wind as a crucial parameter to decide where the fire is spreading. Wind is therefore clever to use in the interpretation process. To make it a useful parameter the kernel in pyXGPR has been edited. The kernel contains all the different covariance functions and noise functions. The modification has been done in the function which measures the squared distance between two points. The squared distance is measured by creating two matrices for each dimension. In the fire interpret these dimensions are x, y and t. One matrix containing all sensor data is deducted from a matrix containing all the uncalculated data. The dimensional result is multiplied with itself and added to a distance matrix. All dimensional results are added to a distance matrix. 

\begin{eqnarray}
\theta = cos^{-1}\left(\dfrac{vector_{ij} \times windVector}{vector_{ij} \times windVector } \right) 
\label{eq:int-angle}
\end{eqnarray}
\begin{eqnarray}
weight = \dfrac{\theta}{\pi}
\label{eq:weight}
\end{eqnarray}

\begin{eqnarray}
\begin{bmatrix} distance_{11} & distance_{1j} \\ distance_{i1} & distance_{ij} \end{bmatrix} \times 
\begin{bmatrix} weight_{11} & weight_{1j} \\ weight_{i1} & weight_{ij} \end{bmatrix}
\label{eq:hadamard}
\end{eqnarray}

Before these calculations the interpreter implementations in this function creates a weight matrix which has the same column and row values as the distance matrix. This weight is calculated with the forumula in \ref{eq:weight}. The Hadamard product (see \ref{eq:hadamard}) of these two matrices is returned as the new distance matrix. 
\\\\
Where $ vector_{ij} $ is the vector from sensor position $i$ to uncalculated position $j$. The weight is further normalized to better fit the wind. If $ vector_{ij} $ between $ 90^{\circ} $ and $ 180^{\circ} $ weight will be larger than 1 and less than $ 45^{\circ} $ weight will be less than 1. The values in the weight matrix is multiplied with the values in the distance matrix. The $ vector_{ij} $ the weight matrix is multiplied with $ vector_{ij} $ the distance matrix. This modifies the distance matrix where some values are reduces and some are increased, determined by their vector direction. The correlation of cells looks at the distance between them. Therefore decreased distance gives a cell a higher probability of being on fire. It should be mentioned that sensors which does not detect fire will have a weight of 1.d

\begin{figure}[here]
  \centering
      \includegraphics[width=1.0\textwidth]{solution/graphics/wind-problem.png}
  \caption{NEED TO SET A USEFUL CAPTION. NEED TO SET A USEFUL CAPTION}
  \label{fig:wind-problem}
\end{figure}

The implementation of wind is working, but has some drawbacks. The first image above illustrates how the fire is predicted when the wind is blowing from the east. The spread starts where the sensor is located and continuous in the shape of a triangle towards west. This looks good in the first image, but the second image illustrates the drawbacks. The wind is still blowing from the east and the two sensors detecting fire are on a line. The right tips of the predicted fire is where the sensors are located. The blue squares is used to highlight where there should be predicted fire. This situation occurs because the blue squares are closest to the second sensor. The vector which goes from the second sensor to the any of blue square has an angle which is more than $ 45^{\circ} $ when compared to the wind vector which is $ \left[-1,0\right] $. Problems with predicting the middle part of the fire can occur when the the size of the predicted fire is becoming quite large. To solve both these problems there has been developed two solutions which have been tested with varying results. The first solution looks at the correlation between all sensors sensing fire while the second solution uses some techniques found in graphical programming. The best would be to avoid these and instead approach these problems with dynamic parameters. As the predicted fire grew larger the parameters changed. But problems arises with such a solution as well. The edge of the fire is more likely to be smudged out and it would in some cases predict fire in areas where there were no fire.
\\\\
Burning sensor correlation was a way to solve the problem when applying wind. It creates a list of all sensors which are sensing fire. Each element in the list has vectors to all other sensors communicating fire. In \ref{fig:burning-sensor-correlation} the green triangle is an uncalculated point while the numbered red circles are sensors sensing fire. There are vectors from sensor 1 to all other sensors sensing fire and a vector to the green triangle. This is an illustration of one of the entries in this list. The basis for this theory is that there is probably actual fire between the sensors. The blue vector compares direction with all the black vectors, see \ref{eq:int-angle}. It finds the black vector which has the most similar direction and check if it is within a certain threshold. If this comes out positive the distance for this point to sensor one will be multiplied with a number less than one, else it would be multiplied with one.
\begin{figure}[here]
  \centering
      \includegraphics[width=0.6\textwidth]{solution/graphics/burning-sensor-correlation.png}
  \caption{NEED TO SET A USEFUL CAPTION. NEED TO SET A USEFUL CAPTION}
  \label{fig:burning-sensor-correlation}
\end{figure}
SLOW SLOW SLOW SLOW SLOW SLOW SLOW SLOW SLOW SLOW SLOW SLOW SLOW SLOW SLOW 
This approach has a fault. When a situation like \ref{fig:wind-problem} (second image) occurs it would predict fire against the wind. The burning sensor correlation would neutralize the added wind to such a degree that it was removed.
\\\\
In the second approach to solve the wind and filling issues, components from computer graphics were used. The outer boundary of the sensors sensing fire is chosen. The shape will be a convex polygon. Uncalculated cells within this polygon will have a higher probability of being on fire in the prediction. As in the burning sensor correlation implementation it is assumed its a higher probability of being fire between sensors sensing fire.
\begin{figure}[here]
  \centering
      \includegraphics[width=0.6\textwidth]{solution/graphics/graphical-boundary.png}
  \caption{NEED TO SET A USEFUL CAPTION. NEED TO SET A USEFUL CAPTION}
  \label{fig:graphical-boundary}
\end{figure}
To locate all the outer cells Graham's scan \cite{graham} was applied. When these points were found Bresenham's line algorithm was used to find the in between cells. The result was all points which the lines covered in figure \ref{fig:graphical-boundary} was retrieved. After this a scan line fill algorithm was used to find all the cells within the polygon. These steps resulted in a list of cell positions. This list was used when calculating the wind in accordance to wind direction. If the cell to be evaluated was inside this polygon the weight was multiplied with a number lower than one.
\\\\
Before the interpreter posts the calculated data to the visualizer it converts the mean values to discrete values, more precisely fire or not fire. The fire threshold can be difficult to determine as time progress. The input values for sensors sensing fire is $0 $ to $10$, while sensors which are not sensing fire is converted from $ 0 $ to $ -1 $. The reason for the conversion is to get some more distance between fire and not fire. The fire threshold is set low to make sure all the realistic fire is covered.
%\include{./performance/performance}
%\chapter{Discussion}
\section{Results}
The results are divided into two groups, simple simulation and advanced simulation.
\subsection{Results with simple simulation}
The simple simulation is characterized by sensors being spread evenly throughout the map. The rule is that a sensor cannot touch another sensor. The figure below illustrates the success of the fire interpreter. The light green color is where the actual fire is and where the interpreter predicted it to be. Therefore the color of success. The dark green color is area where the interpreter thought it would be fire, but was not. The red parts are the actual fire which the interpreter did not predict. When using the simple simulator each sensor can only sense its closest neighbours. 
\begin{figure}[here]
  \centering
      \includegraphics[width=0.5\textwidth]{discussion/graphics/results-simple-compare.png}
  \caption{The intense green is where the fire interpreter predicted correctly. The dark green is where the fire interpreter thought it was fire, but it was not. Red dots are where there actually was fire, but the predictor was unable to predict it.}
  \label{fig:simple-results1}
\end{figure}


\subsection{Results with advanced simulation}
The advanced simulation mimics the effect of humidity and wind has on a fire. The sensors are spread more randomly than in the simple simulation. They can also sense with a larger range. The default is two cells. In these tests wind were enabled in the interpreter to better cope with the wind in the simulation. The illustration below has the real fire on the left together with the sensor (yellow dot) and on the right interpreter prediction.

\begin{figure}[here]
  \centering
      \includegraphics[width=0.5\textwidth]{discussion/graphics/results-simple-compare.png}
  \caption{The intense green is where the fire interpreter predicted correctly. The dark green is where the fire interpreter thought it was fire, but it was not. Red dots are where there actually was fire, but the predictor was unable to predict it.}
  \label{fig:advanced-results1}
\end{figure}
\section {Practical Issues}
There are several practical issues that arises if this system should be put into motion. Collecting sensor data from a mobile phone and sending it to the crisis mapping system creates huge security and privacy issues. Personal data, like location and private conversations, may be useful in mapping a crisis, nevertheless the program should endeavour to protect the user and not disclose any information that is not relevant. In addition, the collected data should be transferred as fast as possible, over a secure connection, from the phone to the crisis mapping system to allow the program to react to new sensor readings as quickly as possible.

The system will always rely on sensors and people carrying a mobile phone capable of sensing. People will spectate on fires, but they will do it outside range of the heat and smoke. If it is blowing in one direction, the smoke and heat will drive people away from one side of the fire, meaning that there will not be sensor readings around the fire after they have moved away.

The mobile phone sensing scheme as with most other schemes, there are limitations to what is possible to achieve. Firstly, the crisis mapping system cannot account for human behavior. When a fire crisis occurs, people need to have a mobile phone with sensors available at that time, and people need to keep the mobile phone sensors close to the crisis situation. The further away the reading is made from the fire, the lower heat and smoke intensity would be read. Even if people would spectate the fire, they would do so away from harmful intensity. In all, this means that there is less good data to work with.
\section {Functional Issues}

We chose to use Gaussian Processes as it is an excelent method for predicting values and it learns by experience. In addition, our supervisor have experience with Gaussian Processes, and helped us getting started with this method of predicting. Since the group had experience with python, and there are some good python libraries for Gaussian, it became the programming language of choice to use for the interpreter.

Our simulation on the other hand was written in JavaScript. The reason for this was that JavaScript is a good choice when one writes graphical user interfaces. Additionally, the group has experience using this language, hence it became the best option. Another challenge we needed to overcome was that we used two different languages to create the solution. We overcame this obstacle by running the interpreter (the Gaussian Processes) on a web server, and then letting our simulation be a web client. The simulation and the interpreter are able to communicate by using a common web interface. Furthermore, one can switch out one of the components and replace it with something that have the same web interface if it became necessary.
%\chapter{Conclusion}

The goal of this project was to help the users determine the best course of action during an evolving crisis. As crisis mapping includes a fairly broad range of crisis situations the goal was to create a prototype crisis mapping system that could detect and map fire using mobile phone sensors. This system were to exclusively use simulated data as setting up mobile phone sensors to properly detect a fire was outside of the project scope. To reach the project goals the system would utilize two grids side by side, one containing the simulated crisis and the other would visualize the data predicted by the Gaussian Processes.

In the end the project met the requirements. In a simple environment the fire interpreter produces good results while in the advanced simulation the fire interpreter produces an accurate shape, but the size of the estimated fire is a little larger. When more environmental factors are introduced to the system the results are less accurate, however the overall performance is quite good.

While the system itself is not ready for deployment it provides insight into the problem and a starting point for future work. The project was never meant to become a finished product. The system is a  proof-of-concept that demonstrates the validity of using Gaussian Processes to predict fire spread. The system provides an easy way of estimating results and as such provides a good framework for future work. There are a few improvements that could be made to advance the current solution.

Firstly the simulation would need to implement a large variety of different squares. In the real world fire is hard to predict as it is affected by several factors. To properly simulate a real fire the simulation would need to simulate a real geographical area with houses, ocean, rivers and so on. Secondly fire would need to behave according to the current understanding of fire and fire movement. I.e., fire moves faster uphill than downhill, there are different types and shapes of fire etc.

Thirdly the simulated sensors move randomly in the current version of the system. To improve upon this it could be beneficial to look into human movement during a crisis and have the sensors move accordingly. Finally, if possible the system should receive and interpret real data from real sensors. This would naturally be the best solution, however the system would likely need to go through several versions of the simulator before it could be used to interpret vast amounts of real data.
\chapter{Introduction}

\section{Background}

Crisis Mapping is the process of collecting, analyzing and visualizing crisis data for further crisis management. The goal is to provide better real time decision support to determine the best course of action in a changing or evolving crisis environment. This in turn provide help to crisis victims and save more lives \cite{irevol}.
\\\\
In 2010, a major earthquake crisis struck Haiti. The shock measured 7.0 on the moment magnitude scale (MMS) and was followed by another 52 aftershocks of 4.0 MMS or higher over 12 days. 1 517 000 people and their dependents were affected. People were in need of housing, food and medical attention. Mapping these needs would improve further crisis management \cite{usha}.
\\\\
The needs on Haiti were mapped by Ushahidi\cite{usha}, an open source software. This software collects, analyzes and visualizes data from social media, text messages and more. The Haiti government took advantage of this software and allowed crisis victims and bystanders to report needs to temporary web services and text message services. These services together with Ushahidi provided a crisis map that aided the government to send help where it was needed \cite{usha}.
\\\\
Today, researchers are investigating how to use mobile phone sensors to map crisis situations. Many mobile phones have built-in sensors such as microphones, accelerometers, gyroscopes, barometers and GPS. In theory, these sensors can be used to detect earthquakes, fires, explosions, collisions and other crisis situations by reading abnormal magnitudes or patterns in sound, acceleration, air pressure and temperature. Mobile phones can be used to report this abnormality to a central crisis mapping system and possibly aid further crisis management.
\\\\
\section{Problem Statement}

CISTECH of the University of Agder proposed a basic project about crisis mapping using mobile phones. CISTECH is  Centre for Information Systems Technology in Emergency Preparedness \& Management. Because crisis mapping is a general term and includes a broad range of different crisis situations, we had to narrow down the goal of the project. After consulting with our supervisor, the final goal was to create a prototype crisis mapping system that could detect fire using mobile phone sensors. Fire was chosen specifically because it can change or evolve over time. However, the chosen topic does not come without a set of practical limitations.

Given the project goal, we focused on two main problems that had to be solved. Firstly, fires had to be simulated. There was no way to justify starting real fires to gather this data. Simulating fire is challenging because fire spreads based on several factors such as wind, heat, humidity and other over time. Secondly, mobile phone sensors do only report readings at their location and do not cover burning areas without nearby sensors. Uncovered areas have to be guesstimated.

\section{Literature Review}

\subsection{A Survey of Mobile Phone Sensing}

The field of mobile phone sensing is rapidly expanding \cite{mobsurv}. The accelerometer in a smart phone can determine whether the user is walking, running or standing still. The microphone can collect audio. The light sensors can determine whether the phone is in a pocket or not and the barometer can determine air pressure. In addition data can be combined from different sensors to make assumptions, for instance a phone can sense that a user is walking down a hill and in what direction. Specialized sensors have also been used to measure air pollution and could perhaps be used to detect explosive residue.

There are many issues with properly utilizing and gathering data. First of all a large amount of users to properly test a system is needed. This can be solved by distributing the application via the App stores and providing an incentive for the users to use it. Second there are huge privacy issues with listening in on private conversations using the microphone and tracking the user via the GPS. This can somewhat be fixed by calculating any result on the phone and sending only the results, however this would require more resources from the phone and would likely limit battery time or performance. Some studies have shown that sensing systems have reduced the standby time by significant amounts, vastly affecting the user experience.

Creating a system that produce accurate results may also be slightly problematic as vendors did not foresee the use of their sensors in new applications. The sensor sampling rate varies based on the CPU load and the API and OS does not support complete access to the sensors. However, as the mobile phone sensing field is growing vendors are likely to catch on and provide better sensors and sensor support in future releases.

Moreover, it is difficult to create a program that fully accounts for all possible instances. Explosives used in demolition and terrorism both create the same or similar sound and gun fire while hunting sounds the same as gun fire in a super marked. A phone also needs to know if it is currently inside a container where the sound might be suppressed, in which case the system might become more sensitive to sound. This in turn can be somewhat solved by using light or proximity sensors however there might be cases where the phone is placed in an unusual location and therefore provides inaccurate results.

\subsection{Distributed Perception Networks for Crisis Management}

Distributed Perception Networks (DPN) for Crisis management (2006)\cite{dpn} focuses on the fusion of information gathered from fixed sensors and human perception. The fixed sensors are few and scarcely distributed in an area where a crisis could have devastating consequences. Endsley’s model tells that first comes perception, then comprehension and lastly projection. The paper focuses on combining the perception of humans and the fixed sensors to map the crisis situation. For example when the sensors are picking up dangerous levels of ammonia in the air, the system will use human perception to gather more input. This is done by sending SMS to everyone in that area where the query could be; “Do you smell ammonia?”. By using human perception the crisis management team will get a richer and more detailed map instead of only using the scarcely distributed fixed sensors.

Sensor readings and public perception results in a vast amount of information which needs to be processed. Since it is such an enormous task for manual human labour the paper proposes to use a Bayesian network to find the most probable cause of the crisis. There is a large degree of uncertainty when gathering huge amounts of information from people and possibly faulty sensors. The DPN architecture represents a multi agent system which means it can comprehend large problems in an efficient manner. A set of nodes is processed by one agent before they send it to their father node. The paper does not suggest how to calculate how toxic the air will become or where it will spread.  

\subsection{Video about how Crisis Mapping Helped During the Haiti Earthquake}

When the earthquake hit Haiti in 2010, there was an initiative to map where help was needed \cite{usha}. This increased the efficiency of the rescue personnel. In addition, no emergency call centers were operational directly after the earthquake. Resulting in the crisis map to become an invaluable tool for organizing the aid. People in need of help would send a text message signifying their location and situation, operators would then note what help was needed and place the note on the map showing the location. Organisations could then use this map to deploy personnel in areas where they were needed the most. In addition, any volunteers could view the map and help if they were able.

One thing to note about this system was that messages were written in different languages and then translated to English. Most of the translation were done with assistance from the general public, this highlighted the positives of a crisis map which is open to the public.

By using a map and plot what is happening in the different areas, organisations that are trying to help may send their resources to the correct locations at the right time. Organisations may also inform people where dangerous locations are during the crisis and guide the people in that area to a safe location. In addition, they may plan their next move as they observe the evolving crisis. People that are able to see this crisis map may see the locations that are dangerous, for example they can then use the map to spot a burning building and head in a different direction. 

\subsection{A Real-time Disaster Situation Mapping System for University Campuses}

A real-time disaster situation mapping (RDSM) system is a web-based system which acts as a social media. The system helps its users and the disaster management team during a crisis situation by giving the disaster management team a clear overview of the situation and the users can use the system to acquire their location and the nearest exit.

The RDSM system is based on some conditions such as disaster type, place of use and targeted users. The main focus of the system is to protect people during and immediately after a disaster based on the assumptions that the earthquake has a minimal impact on the university buildings. In the event that the university buildings are severely damaged then important equipment such as servers, Wireless LAN, etc. may be out of commission and the system may end up not working properly. The mobile devices, such as computers or phones, are assumed to be in use and connected to the servers during the disaster situation. The RDSM system can be accessed inside the university as well as in the evacuation office by any employee with sufficiently high authority level.

The RDSM system has two major parts, called situation gathering sub system (SGS) and the situation mapping sub system (SMS) \cite{springlink}. The SGS allows people in the campus to send situation information about their location, such as degree of longitude, latitude and name of the place at the time of the disaster, using their mobile devices.The location information can be identified by manual entry as well as by the automatic acquisition using Place Engine. Place Engine is a service where users can send location information, even though they do not recognize their locations, by utilizing Wi-Fi devices. Some requirements have to be satisfied in order to set up the Place Engine for automatic data acquisition of location information. First, the Wireless LAN access point data must be saved in the Place Engine database. Secondly, at least one access point must be reachable from the users  location. Next, access to the Place Engine server must be feasible. Finally, the users’ mobile device(s) (PC or phone) must be equipped with Wireless LAN (Wi-Fi) equipment and the client software must be installed on those mobile device(s) \cite{springlink}. The users can send situation information from the situation data entry screen by accessing the SGS. Speech to text transformation is achieved by using “w3voice”, which is a development kit for voice-enabled web applications \cite{springlink}. Therefore the informers can send voice data, text data as well as video via the disaster situation entry screen.The entered situation data is stored in the situation database. 

The SMS reads data from the situation data base updated by SGS and it will replicate that data onto maps sent by the users. The more detailed information of the disaster situation is shown by the time series on the map screen. The users can also access the voice or image data  in order to get a clear and accurate information about the crisis situation.Situation data that are described by text data are rendered on this “png” map in SMS. In order to show the disaster affected areas (points) on the map, Google maps are used to get the degrees of latitude and longitude of two different points. “By calculating the distance between user’s location and the point obtained from Google maps, users’ location is identified from the reduced scale distance that is started from two different points” \cite{springlink}.

\section{Solution Approach}

In this project, we simulated our crisis mapping system using two grids. Both grids represented the same area (any area) divided into equally sized squared cells. In the first grid, we let fire and mobile phone sensors spread freely over the cells. In the second grid, we estimated the fire situation based on mobile sensor readings from the first grid using Gaussian Processes Regression (GPR). Having the actual situation put next to the estimated situation, we could easily verify the strength and weaknesses of both the crisis mapping system and concept in terms of similarity and differences.

\section{Report Outline}

Chapter one introduces crisis mapping, what problem this project is trying to solve and the solution approach. In addition the chapter contains literature reviews of relevant papers. The second chapter of the report contains the theoretical background which the project is based on. The fire theory is used in the implementation of the fire simulation and the theory on Gaussian Processes is used in the implementation of the interpreter.

Chapter tree describes the project requirements and limitations. Additionally it outlines what the simulation and interpreter were expected to do, how the expected functions were implemented and if our way of implementing the features were the correct one. The fourth chapter contains the discussion which aims to present the results, explain why the solution approach was chosen over other alternative solutions and how well the solution satisfied the requirements.

Finally, chapter five contains the conclusion. It endeavours to determine whether the results makes a difference to existing solutions and what the benefits are for system users. Additionally it contains proposed next steps if the project were to be continued.
\chapter{Theory}
\section{Gaussian Processes for Regression}
Mathematical solutions such as linear regression are often used to predict unknown values based on similar values. Depending on input data, output accuracy and simplicity, mathematicians are encouraged to use several other methods as well.
Gaussian Process Regression (GPR) is one of the popular methods has been used due to several reasons such as:
\begin{itemize}
\item Very accurate output
\item Less number of variables
\item Learning by experience
\end{itemize}
Correct predictions must be based on the correct assumptions. Even though GPR is  learning from experience, it is not “free form” which means it is not automatically generated\cite{gausreg}. Other techniques, such as “squared exponential”, can be used when the users are unable to do even basic assumptions.
\\\\
It is acceptable to assume that each observation point has data distribution similar to normal distribution. Therefore, it is needed  to calculate the mean value and variance for all predictions.
\begin{figure}[here]
  \centering
      \includegraphics[width=0.9\textwidth]{theory/graphics/normal-distribution.png}
  \caption{Normal Distribution curves\cite{normal-dist}. }
  \label{fig:normal-distribution}
\end{figure}
\subsection{Covariance Function}
In general, users can assume that “partner GP’s mean is zero” everywhere\cite{simple-covariance}. Therefore, two similar cases are related to each other by a covariance function $ k(x-x') $. Among the various options “squared exponential” is one of the popular choices for the covariance function.
\begin{equation}
k(x,x^{'})=\sigma_{f}^{2}exp\left[ \dfrac{-(x-x^{'})^{2}}{2l^{2}} \right]
\label{eq:simple-covariance}
\end{equation}
\begin{eqnarray}
 x - x^{'} &=& Distance\hspace{7pt} between\hspace{7pt} observation\hspace{7pt} data    \nonumber \\
  l &=& length\hspace{7pt} parameter \nonumber \\
  \sigma_{f}^{2} &=& Maximum\hspace{7pt} allowable\hspace{7pt} covariance
\end{eqnarray}
\\\\
To predict accurate data for a smooth curve, input (or observed) data from the neighbours must be identical. By studying the above equation, it can understood that $ \sigma_{f}^{2} $ is increased for functions which cover a wide area through the y-axis. When the $ (x-x') $value becomes larger, $k(x-x')$ approaches zero depending on the length of parameter $ l $.
\\\\
The effects of the length parameter $l$, can be explained as follows: Lets assume $l=0.1$ for a particular prediction and graph  $f(x)VS x$, which is shown in figure \ref{fig:length-parameter}. It is clear that the curves are not aligned\cite{intro-to-gpr}.
\begin{figure}[here]
  \centering
      \includegraphics[width=0.65\textwidth]{theory/graphics/effects-of-l.png}
  \caption{When $l = 0.1$. \cite{length-parameter}. }
  \label{fig:length-parameter}
\end{figure}
\\\\
Then change $l=0.3$ and evaluate the results. This is shown in figure \ref{fig:length-parameter-03}. It is clear that the curves are more aligned here than in figure \ref{fig:length-parameter}.
\begin{figure}[here]
  \centering
      \includegraphics[width=0.65\textwidth]{theory/graphics/effects-of-l-03.png}
  \caption{When $l = 0.3$. \cite{length-parameter}. }
  \label{fig:length-parameter-03}
\end{figure}
\\\\
Since the curves are not perfectly aligned, length parameter $l$, is increased further and result is evaluated. In figure \ref{fig:length-parameter-05} the $l$ value is set to $0.5$ and evaluate the results. Here one can observe that the curves are perfectly aligned \cite{length-parameter}.
\begin{figure}[here]
  \centering
      \includegraphics[width=0.65\textwidth]{theory/graphics/effects-of-l-05.png}
  \caption{When $l = 0.5$. \cite{length-parameter}. }
  \label{fig:length-parameter-05}
\end{figure}
\\\\
Apart from this, function consists of additional part to represent errors. This is an important part, when GPR is used to predict practical events. (e.g. environment temperature can be vary, due to external factors such as wind ).
\begin{equation}
k(x,x^{'})=\sigma_{f}^{2}exp\left[ \dfrac{-(x-x^{'})^{2}}{2l^{2}} \right] + \sigma_{n}^{2}\delta(x,x^{'})
\label{eq:referanceName}
\end{equation}
\\\\
However, people preferred to keep $ \sigma_{n}^{2} $ separately and work on covariance and error to simplify the calculation.

\subsection{Method of Calculation}
The covariance value, $ k(x,x') $ can be calculated with each observed data in respect to each other observations. Values can be represented by a metric. By assuming that five observations have been conducted, $ x1, x2, x3, x4, x5 $. $k(x,x')$ value for those observations can be written as follows:
\begin{eqnarray}
K = 
\begin{bmatrix} 
k(x_{1}x_{1}) & k(x_{1}x_{2} & k(x_{1}x_{3} & k(x_{1}x_{4} & k(x_{1}x_{5} \\ k(x_{2}x_{1}) & k(x_{2}x_{2} & k(x_{2}x_{3} & k(x_{2}x_{4} & k(x_{2}x_{5} \\
k(x_{3}x_{1}) & k(x_{3}x_{2} & k(x_{3}x_{3} & k(x_{3}x_{4} & k(x_{3}x_{5} \\ k(x_{4}x_{1}) & k(x_{4}x_{2} & k(x_{4}x_{3} & k(x_{4}x_{4} & k(x_{4}x_{5} \\
k(x_{5}x_{1}) & k(x_{5}x_{2} & k(x_{5}x_{3} & k(x_{5}x_{4} & k(x_{5}x_{5} \\ \end{bmatrix}
\label{eq:k-matrix}
\end{eqnarray}
\\\\
This is a symmetric metric, $k(x1,x2)$ and $k(x2,x1)$  where both elements have the same values. All diagonal elements are equal and show the highest $k(x,x')$ value since $ \dfrac{-(x-x^{'})^{2}}{2l^{2}} $  becomes zero.
\\\\
When the position is moving away from the diagonal line of the matrix, the value of $k(x_{n},x_{m})$ goes towards zero. In other words, if the matrix contain higher number of rows and columns, the values of positions such as far away from the diagonal are almost equal to zero, while diagonal values shows maximum covariance value.
\\\\
While the $K$ matrix gives the covariance values for observed data points the $K*$ matrix gives the covariance values for considered points compared to other observed points. Let’s make the matrix $K_{*}$ relative to the above five positions.
\begin{equation}
K_{*} = [k(x_{*},x_{1}) \hspace{10pt} k(x_{*},x_{2}) \hspace{10pt} k(x_{*},x_{3}) \hspace{10pt} k(x_{*},x_{4}) \hspace{10pt} k(x_{*},x_{5})]
\label{eq:referanceName}
\end{equation}
\\\\
$K_{**}= k(x_{*} x_{*})$ also can be found same way as it is directly given by $k(x,x')=\sigma_{f}^{2}$. From the above values, mean and variance values for a given $y_{*}$ can be found in the following way:
\begin{eqnarray}
 \overline{y_{*}} &=& K_{*}K^{-1} y    \nonumber \\
 var(y_{*})&=& K_{**}K_{*}K^{-1}K_{*}^{T}
\end{eqnarray}
\begin{eqnarray}
y = 
\begin{bmatrix} 
y_{1} \\ y_{2} \\ y_{3} \\ y_{4} \\ y_{5} \\ \end{bmatrix}
\label{eq:k-matrix}
\end{eqnarray}
\\\\
When finding the variance, a suitable confidence level must be selected according to the prediction (e.g. 95\% confidence level).
\subsection{Prediction of Values Using Gaussian Process Regression}
The objective is to predict values at a particular point by interpreting other observed data. Imagine the following situation: The area highlighted in gray colour has sensors to observe the values and one would want to predict the value on red colour(3,3) in figure \ref{fig:value-matrix}.
\begin{figure}[here]
  \centering
      \includegraphics[width=0.65\textwidth]{theory/graphics/value-matrix.png}
  \caption{ Value matrix. }
  \label{fig:value-matrix}
\end{figure}
\\\\
Since it is possible to find the distance between each cells, it is possible to define a covariance function using observed data (given from sensors). $ \sigma_{f}^{2} $ and $ l $ (length parameter) can be determined by using the observed data. Thereafter, $ K_{**} $, $ K_{*} $, $ K $ matrices can be determined. This will give the values for mean and variance. Finally, it is possible to plot the graph and this will give the predicted values for any positions within the considered area.

\chapter{Solution}
\section{Fire and Sensor Simulation}
\section{Fire Interpreter}

The fire interpreter's job is to receive input from the simulator and 
calculate values for the received cells. The calculated values are then posted to the visualizer. The advanced calculations are done by the library pyXGPR. This is a Gaussian Process Regression library implemented with Python. It produces a mean and a variance when used correctly. The first input parameter \textbf{X} is a list of points which tells where the training data is located. Another parameter \textbf{Y} contains the values to the training data. The last interpreter generated parameter \textbf{x star} contains the points where we want to find the mean and the variance. In addition to these parameters the library needs to be told what covariance functions pyXGPR should use to calculate the correlation between the cells in \textbf{X}, \textbf{Y} and \textbf{x star}. There is also added parameter values to these functions.
\\\\
The most basic use of pyXGPR is one dimensional (line regression) where \textbf{X} is the location and \textbf{Y} is the value. The fire interpreter uses regression in three dimensions where \textbf{X} and \textbf{Y} are the map coordinates and an additional parameter \textbf{t} is for time. \textbf{t} is necessary to save earlier sensor data which later are utilized in calculations. It should also be mentioned that before this implementation, this was done by saving the best data. Best data is to be understood as the data which has the lowest variance. Data with lower variance would be applied to the saved map. This hack and the implementation of t is done because previous sensor data is important as long as they are weighted less than the newest sensor data. As time increases there will be sensor data covering most of the map, but the old sensor data will have less weight and thus giving new sensor data the opportunity to be taken into account. Figure \ref{fig:timeElapse} illustrates this.


\begin{figure}[here]
  \centering
      \includegraphics[width=0.5\textwidth]{solution/graphics/timeElapse.png}
  \caption{NEED TO SET A USEFUL CAPTION. NEED TO SET A USEFUL CAPTION}
  \label{fig:timeElapse}
\end{figure}


The simulation map is 71 X 71 squares. GPR calculates the correlation between all points. This means pyXGPR creates huge matrices. As t increases the the matrices grow larger and thus the execution time rises. The first version of the program used more than 6 minutes to run on a normal laptop. To better the performance it was found necessary to locate some squares which did not need to be calculated. Cells containing sensors was therefore removed from \textbf{x star} since the values were already there. With the introduction of t the matrices became larger and execution time rose and once again it was necessary to find some steps to improve the performance. For a cell to be added to \textbf{x star} it had to be close to sensor which communicated fire. This was done by finding the euclidean distance between all cells which had no value and all sensors which communicated fire. This reduced the number of points in \textbf{x star} significantly. As time increases Another measure taken to reduce the execution time was to save cells which was calculated to be on fire. From these the euclidean distance to all points were calculated and if they were within a certain distance they would be added to \textbf{x star}.
\\\\
The simulation use wind as a crucial parameter to decide where the fire is spreading. Wind is therefore clever to use in the interpretation process. To make it a useful parameter the kernel in pyXGPR has been edited. The kernel contains all the different covariance functions and noise functions. The modification has been done in the function which measures the squared distance between two points. The squared distance is measured by creating two matrices for each dimension. In the fire interpret these dimensions are x, y and t. One matrix containing all sensor data is deducted from a matrix containing all the uncalculated data. The dimensional result is multiplied with itself and added to a distance matrix. All dimensional results are added to a distance matrix. 

\begin{eqnarray}
\theta = cos^{-1}\left(\dfrac{vector_{ij} \times windVector}{vector_{ij} \times windVector } \right) 
\label{eq:int-angle}
\end{eqnarray}
\begin{eqnarray}
weight = \dfrac{\theta}{\pi}
\label{eq:weight}
\end{eqnarray}

\begin{eqnarray}
\begin{bmatrix} distance_{11} & distance_{1j} \\ distance_{i1} & distance_{ij} \end{bmatrix} \times 
\begin{bmatrix} weight_{11} & weight_{1j} \\ weight_{i1} & weight_{ij} \end{bmatrix}
\label{eq:hadamard}
\end{eqnarray}

Before these calculations the interpreter implementations in this function creates a weight matrix which has the same column and row values as the distance matrix. This weight is calculated with the forumula in \ref{eq:weight}. The Hadamard product (see \ref{eq:hadamard}) of these two matrices is returned as the new distance matrix. 
\\\\
Where $ vector_{ij} $ is the vector from sensor position $i$ to uncalculated position $j$. The weight is further normalized to better fit the wind. If $ vector_{ij} $ between $ 90^{\circ} $ and $ 180^{\circ} $ weight will be larger than 1 and less than $ 45^{\circ} $ weight will be less than 1. The values in the weight matrix is multiplied with the values in the distance matrix. The $ vector_{ij} $ the weight matrix is multiplied with $ vector_{ij} $ the distance matrix. This modifies the distance matrix where some values are reduces and some are increased, determined by their vector direction. The correlation of cells looks at the distance between them. Therefore decreased distance gives a cell a higher probability of being on fire. It should be mentioned that sensors which does not detect fire will have a weight of 1.d

\begin{figure}[here]
  \centering
      \includegraphics[width=1.0\textwidth]{solution/graphics/wind-problem.png}
  \caption{NEED TO SET A USEFUL CAPTION. NEED TO SET A USEFUL CAPTION}
  \label{fig:wind-problem}
\end{figure}

The implementation of wind is working, but has some drawbacks. The first image above illustrates how the fire is predicted when the wind is blowing from the east. The spread starts where the sensor is located and continuous in the shape of a triangle towards west. This looks good in the first image, but the second image illustrates the drawbacks. The wind is still blowing from the east and the two sensors detecting fire are on a line. The right tips of the predicted fire is where the sensors are located. The blue squares is used to highlight where there should be predicted fire. This situation occurs because the blue squares are closest to the second sensor. The vector which goes from the second sensor to the any of blue square has an angle which is more than $ 45^{\circ} $ when compared to the wind vector which is $ \left[-1,0\right] $. Problems with predicting the middle part of the fire can occur when the the size of the predicted fire is becoming quite large. To solve both these problems there has been developed two solutions which have been tested with varying results. The first solution looks at the correlation between all sensors sensing fire while the second solution uses some techniques found in graphical programming. The best would be to avoid these and instead approach these problems with dynamic parameters. As the predicted fire grew larger the parameters changed. But problems arises with such a solution as well. The edge of the fire is more likely to be smudged out and it would in some cases predict fire in areas where there were no fire.
\\\\
Burning sensor correlation was a way to solve the problem when applying wind. It creates a list of all sensors which are sensing fire. Each element in the list has vectors to all other sensors communicating fire. In \ref{fig:burning-sensor-correlation} the green triangle is an uncalculated point while the numbered red circles are sensors sensing fire. There are vectors from sensor 1 to all other sensors sensing fire and a vector to the green triangle. This is an illustration of one of the entries in this list. The basis for this theory is that there is probably actual fire between the sensors. The blue vector compares direction with all the black vectors, see \ref{eq:int-angle}. It finds the black vector which has the most similar direction and check if it is within a certain threshold. If this comes out positive the distance for this point to sensor one will be multiplied with a number less than one, else it would be multiplied with one.
\begin{figure}[here]
  \centering
      \includegraphics[width=0.6\textwidth]{solution/graphics/burning-sensor-correlation.png}
  \caption{NEED TO SET A USEFUL CAPTION. NEED TO SET A USEFUL CAPTION}
  \label{fig:burning-sensor-correlation}
\end{figure}
SLOW SLOW SLOW SLOW SLOW SLOW SLOW SLOW SLOW SLOW SLOW SLOW SLOW SLOW SLOW 
This approach has a fault. When a situation like \ref{fig:wind-problem} (second image) occurs it would predict fire against the wind. The burning sensor correlation would neutralize the added wind to such a degree that it was removed.
\\\\
In the second approach to solve the wind and filling issues, components from computer graphics were used. The outer boundary of the sensors sensing fire is chosen. The shape will be a convex polygon. Uncalculated cells within this polygon will have a higher probability of being on fire in the prediction. As in the burning sensor correlation implementation it is assumed its a higher probability of being fire between sensors sensing fire.
\begin{figure}[here]
  \centering
      \includegraphics[width=0.6\textwidth]{solution/graphics/graphical-boundary.png}
  \caption{NEED TO SET A USEFUL CAPTION. NEED TO SET A USEFUL CAPTION}
  \label{fig:graphical-boundary}
\end{figure}
To locate all the outer cells Graham's scan \cite{graham} was applied. When these points were found Bresenham's line algorithm was used to find the in between cells. The result was all points which the lines covered in figure \ref{fig:graphical-boundary} was retrieved. After this a scan line fill algorithm was used to find all the cells within the polygon. These steps resulted in a list of cell positions. This list was used when calculating the wind in accordance to wind direction. If the cell to be evaluated was inside this polygon the weight was multiplied with a number lower than one.
\\\\
Before the interpreter posts the calculated data to the visualizer it converts the mean values to discrete values, more precisely fire or not fire. The fire threshold can be difficult to determine as time progress. The input values for sensors sensing fire is $0 $ to $10$, while sensors which are not sensing fire is converted from $ 0 $ to $ -1 $. The reason for the conversion is to get some more distance between fire and not fire. The fire threshold is set low to make sure all the realistic fire is covered.
\chapter{Discussion}
\section{Results}
The results are divided into two groups, simple simulation and advanced simulation.
\subsection{Results with simple simulation}
The simple simulation is characterized by sensors being spread evenly throughout the map. The rule is that a sensor cannot touch another sensor. The figure below illustrates the success of the fire interpreter. The light green color is where the actual fire is and where the interpreter predicted it to be. Therefore the color of success. The dark green color is area where the interpreter thought it would be fire, but was not. The red parts are the actual fire which the interpreter did not predict. When using the simple simulator each sensor can only sense its closest neighbours. 
\begin{figure}[here]
  \centering
      \includegraphics[width=0.5\textwidth]{discussion/graphics/results-simple-compare.png}
  \caption{The intense green is where the fire interpreter predicted correctly. The dark green is where the fire interpreter thought it was fire, but it was not. Red dots are where there actually was fire, but the predictor was unable to predict it.}
  \label{fig:simple-results1}
\end{figure}


\subsection{Results with advanced simulation}
The advanced simulation mimics the effect of humidity and wind has on a fire. The sensors are spread more randomly than in the simple simulation. They can also sense with a larger range. The default is two cells. In these tests wind were enabled in the interpreter to better cope with the wind in the simulation. The illustration below has the real fire on the left together with the sensor (yellow dot) and on the right interpreter prediction.

\begin{figure}[here]
  \centering
      \includegraphics[width=0.5\textwidth]{discussion/graphics/results-simple-compare.png}
  \caption{The intense green is where the fire interpreter predicted correctly. The dark green is where the fire interpreter thought it was fire, but it was not. Red dots are where there actually was fire, but the predictor was unable to predict it.}
  \label{fig:advanced-results1}
\end{figure}
\section {Practical Issues}
There are several practical issues that arises if this system should be put into motion. Collecting sensor data from a mobile phone and sending it to the crisis mapping system creates huge security and privacy issues. Personal data, like location and private conversations, may be useful in mapping a crisis, nevertheless the program should endeavour to protect the user and not disclose any information that is not relevant. In addition, the collected data should be transferred as fast as possible, over a secure connection, from the phone to the crisis mapping system to allow the program to react to new sensor readings as quickly as possible.

The system will always rely on sensors and people carrying a mobile phone capable of sensing. People will spectate on fires, but they will do it outside range of the heat and smoke. If it is blowing in one direction, the smoke and heat will drive people away from one side of the fire, meaning that there will not be sensor readings around the fire after they have moved away.

The mobile phone sensing scheme as with most other schemes, there are limitations to what is possible to achieve. Firstly, the crisis mapping system cannot account for human behavior. When a fire crisis occurs, people need to have a mobile phone with sensors available at that time, and people need to keep the mobile phone sensors close to the crisis situation. The further away the reading is made from the fire, the lower heat and smoke intensity would be read. Even if people would spectate the fire, they would do so away from harmful intensity. In all, this means that there is less good data to work with.
\section {Functional Issues}

We chose to use Gaussian Processes as it is an excelent method for predicting values and it learns by experience. In addition, our supervisor have experience with Gaussian Processes, and helped us getting started with this method of predicting. Since the group had experience with python, and there are some good python libraries for Gaussian, it became the programming language of choice to use for the interpreter.

Our simulation on the other hand was written in JavaScript. The reason for this was that JavaScript is a good choice when one writes graphical user interfaces. Additionally, the group has experience using this language, hence it became the best option. Another challenge we needed to overcome was that we used two different languages to create the solution. We overcame this obstacle by running the interpreter (the Gaussian Processes) on a web server, and then letting our simulation be a web client. The simulation and the interpreter are able to communicate by using a common web interface. Furthermore, one can switch out one of the components and replace it with something that have the same web interface if it became necessary.
\chapter{Conclusion}

The goal of this project was to help the users determine the best course of action during an evolving crisis. As crisis mapping includes a fairly broad range of crisis situations the goal was to create a prototype crisis mapping system that could detect and map fire using mobile phone sensors. This system were to exclusively use simulated data as setting up mobile phone sensors to properly detect a fire was outside of the project scope. To reach the project goals the system would utilize two grids side by side, one containing the simulated crisis and the other would visualize the data predicted by the Gaussian Processes.

In the end the project met the requirements. In a simple environment the fire interpreter produces good results while in the advanced simulation the fire interpreter produces an accurate shape, but the size of the estimated fire is a little larger. When more environmental factors are introduced to the system the results are less accurate, however the overall performance is quite good.

While the system itself is not ready for deployment it provides insight into the problem and a starting point for future work. The project was never meant to become a finished product. The system is a  proof-of-concept that demonstrates the validity of using Gaussian Processes to predict fire spread. The system provides an easy way of estimating results and as such provides a good framework for future work. There are a few improvements that could be made to advance the current solution.

Firstly the simulation would need to implement a large variety of different squares. In the real world fire is hard to predict as it is affected by several factors. To properly simulate a real fire the simulation would need to simulate a real geographical area with houses, ocean, rivers and so on. Secondly fire would need to behave according to the current understanding of fire and fire movement. I.e., fire moves faster uphill than downhill, there are different types and shapes of fire etc.

Thirdly the simulated sensors move randomly in the current version of the system. To improve upon this it could be beneficial to look into human movement during a crisis and have the sensors move accordingly. Finally, if possible the system should receive and interpret real data from real sensors. This would naturally be the best solution, however the system would likely need to go through several versions of the simulator before it could be used to interpret vast amounts of real data.

\clearpage
\addcontentsline{toc}{chapter}{Acknowledgments}
\chapter*{Acknowledgments}
\vspace{1.0in}
We would like to thank our supervisors Ole-Christoffer Granmo for his constructive feedback that has led to progress in times when the project was at a stand still. \\\\

{University of Agder, 2011}\\
\newpage

\addcontentsline{toc}{chapter}{References}

\begin{thebibliography}{4}
\bibitem{graham} Alejo Hausner, CS Department, Princeton University
 \emph{http://www.cs.princeton.edu/courses/archive/spr10/cos226/demo/ah/GrahamScan.html (2011).} Prentice Hall
 
\bibitem{hadamard} Teknomo K. Hadamard Product. 2011. Available from \url{http://people.revoledu.com/kardi/tutorial/LinearAlgebra/HadamardProduct.html} 

\bibitem{bresenham} Flanagan C. The Bresenham Line-Drawing Algorithm. 2011. Available from \url{http://www.cs.helsinki.fi/group/goa/mallinnus/lines/bresenh.html} 

\bibitem{normal-dist} Inductiveload. Normal Distribution: curve. 2008. Available from \url{http://en.wikipedia.org/wiki/File:Normal_Distribution_PDF.svg} 
\bibitem{length-parameter} Wallach H. M. Introduction to Gaussian Process Regression: curve. 2005. Available from \url{http://www.cs.umass.edu/~wallach/talks/gp_intro.pdf} 

\bibitem{simple-covariance} Ebden M. Gaussian Processes for Regression: A Quick Introduction. aug 2008. Available from \url{http://www.robots.ox.ac.uk/~mebden/reports/GPtutorial.pdf} 


\bibitem{intro-to-gpr} Wallach H. M. Introduction to Gaussian Process Regression. jan 25, 2005. Available from \url{http://www.cs.umass.edu/~wallach/talks/gp_intro.pdf} 

\bibitem{fire-tirangleimage} Wikipedia,       \url{http://upload.wikimedia.org/wikipedia/commons/2/20/Fire_triangle.svg}, des 13, 2011

\bibitem{irevol} Patrick Meier, \underline{iRevolution} \url{http://irevolution.net/2011/01/20/what-is-crisis-mapping/} Posted on:jan 20, 2011, Visited on: des 14, 2011

\bibitem{mobsurv} Nicholas D. Lane, Emiliano Miluzzo, Hong Lu, Daniel Peebles, Tanzeem Choudhury,
and Andrew T. Campbell, Dartmouth College, \underline{IEEE Communications Magazine} \url{http://ieeexplore.ieee.org/stamp/stamp.jsp?tp=&arnumber=5560598 } Release: September 2010, Visited on: des 14, 2011

\bibitem{filmcrisis} Alison Stewart, "Need to Know", \underline{Crisis Mapping}, Premier date: may 13, 2011, \url{http://gis-techniques.blogspot.com/2011/05/new-way-of-mapping-crisis-mapping.html}

\bibitem{usha} "Ushahidi" \url{http://www.ushahidi.com/About-us} Visited on: des 14, 2011

\bibitem{springlink} Iizuka K ,Iizuka Y,Yoshida K, \emph{A Real-time Disaster Situation Mapping System for University Campuses}, \url{http://www.springerlink.com/content/0302-9743/?k=crisis+mapping} Publish date: 2011 Visited on: des 15, 2011

\bibitem{gitvers} "Version control glossary", \underline{What is version control} \url{What is version control} Visited on: des 14, 2011

\bibitem{gitis} "git the fast vesion control system", \underline{Git is...} \url{http://git-scm.com/}, Visited on: des 14, 2011

\bibitem{gausreg} Ebden M. \underline{Gaussian Processes for Regression: A Quick Introduction} Publish date: August 2008, \url{http://www.robots.ox.ac.uk/~mebden/reports/GPtutorial.pdf}, Visited on: des 15, 2011

\bibitem{intgausreg} Wallach H. M. \underline{Introduction to Gaussian Process Regression}
Publish date: January 25,2005
\url{http://www.cs.umass.edu/~wallach/talks/gp_intro.pdf}, Visited on: des 15, 2011

\bibitem{firebev} "Fire ecology and management", \underline{Fire Behaviour}, \url{http://learnline.cdu.edu.au/units/sbi263/fundamentals/behaviour.html}
Visited on: des 14, 2011, Last modified: aug 09, 2005

\bibitem{firebas} "Natural Resources Canada", \underline{Fire Basics}, \url{http://fire.cfs.nrcan.gc.ca/questions-fire-feu-eng.php} Visited on: des 14, 2011
Last modified: jun 30, 2011

\bibitem{rateofspread} "Knowledge and Application", \underline{Rate of Spread}, \url{http://www.forestencyclopedia.net/p/p478} Visited on: des 14, 2011, Last modified: nov 14, 2008

\bibitem{wildfire} Stephen J. Pyne, Patricia L. Andrews, Richard D. Laven, \emph{Introduction to Wildland fire. 2nd ed.}, John Wiley and Sons. Inc. 1996: p.37
\url{http://books.google.no/books?id=yT6bzpUyFIwC&pg=PA37&dq=Rothermel%E2%80%99s+equation+have+been+the+basis+for+most+of+the+fire+spread+prediction+models&hl=en&ei=j-foTsi9Esj34QSIof33CA&sa=X&oi=book_result&ct=result&redir_esc=y#v=onepage&q=Rothermel%E2%80%99s%20equation%20have%20been%20the%20basis%20for%20most%20of%20the%20fire%20spread%20prediction%20models&f=false}

\bibitem{firemov} "Fire ecology and management in northern Australia", \underline{Topography}, \url{http://learnline.cdu.edu.au/units/sbi263/fundamentals/topography.html}, 
Visited on: des 15, 2011

\bibitem{fireslope} "Unit2: Topographic Influences on Wildland Fire Behavioir" \url{http://deved.meted.ucar.edu/fire/s290/unit2/media/graphics/slopeandfirebehav.jpg} Visited on: des 15, 2011

\bibitem{firetypeimage} "Fire ecology and management in northern Australia", \underline{Fire regime}, \url{http://learnline.cdu.edu.au/units/sbi263/fundamentals/regime.html}, Visited on: des 15, 2011

\bibitem{firetypeimage} "Fire ecology and management in northern Australia", \underline{Climate, vegetation and fire}, \url{http://learnline.cdu.edu.au/units/sbi263/ecology/climate.html}, Visited on: des 15, 2011

\bibitem{fireweather} "Fire ecology and management in northern Australia", \underline{Fire weather}, \url{http://learnline.cdu.edu.au/units/sbi263/fundamentals/weather.html}, Visited on: des 15, 2011

\bibitem{relvhum} National Wildlife Coordinating Group, \emph{FIRE EFFECTS GUIDE
}, \url{http://www.nwcg.gov/pms/RxFire/FEG.pdf}, Last modified: jun 21, 2001, Visited on: des 15, 2011. 

\end{thebibliography}

%\appendix
%	\begin{appendices}
\begin{flushleft}
\label{summary}{\bf Appendix A} Summary from meeting with supervisor Folke Haugland.\\
\label{taxonomy}{\bf Appendix B } Taxonomy example. \\
\label{comparison}{\bf Appendix C} This is the results from our query optimizer versus Solr.\\
\label{sharded}{\bf Appendix D} This is the results from shards on production system. \\
\label{simultaneous}{\bf Appendix E} This is the results from when we tested solo split simultaneous. \\
\label{ssvsns}{\bf Appendix F} This is the results from when we tested solo split versus non split. \\
\label{knotten}{\bf Appendix G} Project test server tests. \\
\label{production}{\bf Appendix H} Production server tests. \\
\label{press}{\bf Appendix I} Press release. \\
\end{flushleft}
\end{appendices}
%	\include{acro}




%\bibliographystyle{IEEEtranS}
%\bibliography{bibliography}

\end{document}
