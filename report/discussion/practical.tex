\section {Practical Issues}
There are several practical issues that arises if this system should be put into motion. Collecting sensor data from a mobile phone and sending it to the crisis mapping system creates huge security and privacy issues. Personal data, like location and private conversations, may be useful in mapping a crisis, nevertheless the program should endeavour to protect the user and not disclose any information that is not relevant. In addition, the collected data should be transferred as fast as possible, over a secure connection, from the phone to the crisis mapping system to allow the program to react to new sensor readings as quickly as possible.

The system will always rely on sensors and people carrying a mobile phone capable of sensing. People will spectate on fires, but they will do it outside range of the heat and smoke. If it is blowing in one direction, the smoke and heat will drive people away from one side of the fire, meaning that there will not be sensor readings around the fire after they have moved away.

The mobile phone sensing scheme as with most other schemes, there are limitations to what is possible to achieve. Firstly, the crisis mapping system cannot account for human behavior. When a fire crisis occurs, people need to have a mobile phone with sensors available at that time, and people need to keep the mobile phone sensors close to the crisis situation. The further away the reading is made from the fire, the lower heat and smoke intensity would be read. Even if people would spectate the fire, they would do so away from harmful intensity. In all, this means that there is less good data to work with.