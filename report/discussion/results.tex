\section{Results}
The results are divided into two groups, simple simulation and advanced simulation. The results addressed in this section are reasonably representative for the average results.
\subsection{Results with simple simulation}
The simple simulation is characterized by sensors being spread relatively even throughout the map. The rule is that a sensor cannot touch another sensor. When using the simple simulator each sensor can only sense its closest neighbours. 
\\\\
Figure \ref{fig:simple-results1} illustrates the success of the fire interpreter. The light green color is where the actual fire is and where the interpreter predicted it to be. Therefore the color of success. The dark green color is area where the interpreter thought it would be fire, but was not. The red parts are the actual fire which the interpreter did not predict. 
\begin{figure}[here]
  \centering
      \includegraphics[width=0.7\textwidth]{discussion/graphics/results-simple-compare.png}
  \caption{The intense green is where the fire interpreter predicted correctly. The dark green is where the fire interpreter thought it was fire, but it was not. Red dots are where there actually was fire, but the predictor was unable to predict it.}
  \label{fig:simple-results1}
\end{figure}
\subsection{Results with advanced simulation}
The advanced simulation mimics the effect of humidity and wind has on a fire. The sensors are spread more randomly than in the simple simulation. They can also sense with a larger range. The default is two cells. The fire interpreter used the wind add-on in the kernel in the successful tests, while it was turned off in the tests which did not have too good results. There are a number of parameters which has been tweaked to get good results. These different parameters makes it more difficult to get an accurate prediction. Figure \ref{fig:wind-advanced-bresenham-large} illustrates the predicted fire on top of the actual fire when the wind is blowing from west.  
\begin{figure}[here]
  \centering
      \includegraphics[width=0.5\textwidth]{discussion/graphics/wind-advanced-bresenham-large.png}
  \caption{The dark green overlay is where the fire interpreter predicted the fire to be. The red covered by the green overlay is where the fire interpreter predicted correctly and the red dots without any overlay is where there was fire, but did not predict.}
  \label{fig:wind-advanced-bresenham-large}
\end{figure}
The prediction is covering a larger area than the actual fire. This is because the sensors are sensing 2 cells away. In figure \ref{fig:wind-advanced-bresenham-large} this can be observed with the right most predicted fire. The two sensors which lays beneath the right most predicted area is sensing fire and therefore the prediction is set from this point. The first cells north and south of this area has also been calculated to be on fire. This is because of Bresenham’s line algorithm which interprets these two cells to be on the outer boundary of the sensors sensing fire. The algorithms from computer graphics is used to make sure there are no holes inside the predicted fire and to prevent wind problem (chapter \ref{wind-problem}).
\begin{figure}[here]
  \centering
      \includegraphics[width=0.5\textwidth]{discussion/graphics/advanced-without-wind-and-bresenham.png}
  \caption{Wind and graphical algorithms disabled in the advanced simulation creates holes and shows a lack wind direction.}
  \label{fig:advanced-without-wind-and-bresenham}
\end{figure}
\\\\
Figure \ref{fig:advanced-without-wind-and-bresenham} illustrates what is happening when wind and the graphical algorithms are disabled. Within the domain of the advanced simulation the results achieved with wind and the graphical algorithms. The prediction with this set up reproduces the shape of the actual fire with a satisfactory result. The area of the shape is larger than the actual fire. The philosophy was always to air on the side of safety. Meaning that the fire interpreter aimed to detect all fires and as such a few false positives were deemed to be acceptable.