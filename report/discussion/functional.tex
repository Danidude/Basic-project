\section {Functional Issues}

We chose to use Gaussian Processes as it is an excelent method for predicting values and it learns by experience. In addition, our supervisor have experience with Gaussian Processes, and helped us getting started with this method of predicting. Since the group had experience with python, and there are some good python libraries for Gaussian, it became the programming language of choice to use for the interpreter.

Our simulation on the other hand was written in JavaScript. The reason for this was that JavaScript is a good choice when one writes graphical user interfaces. Additionally, the group has experience using this language, hence it became the best option. Another challenge we needed to overcome was that we used two different languages to create the solution. We overcame this obstacle by running the interpreter (the Gaussian Processes) on a web server, and then letting our simulation be a web client. The simulation and the interpreter are able to communicate by using a common web interface. Furthermore, one can switch out one of the components and replace it with something that have the same web interface if it became necessary.