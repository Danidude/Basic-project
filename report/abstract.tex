\vspace{2in}
\begin{abstract}
Crisis mapping is a concept to help crisis management teams to determine the best course of action in a changing or evolving crisis situation in order to help people and save more lives. This report describes how mobile phone sensors can be used to measure, interpret and visualize a fire crisis situation. An interactive simulator was especially designed for this project to provide sensor readings affected by environmental parameters such as wind and humidity. These individual readings were used in the Gaussian Processes Regression (GPR) to estimate the area and magnitude of the fire. The kernel function of the GPR was expanded to also take wind into account to the distance vectors.. The overall system performance were visualized by putting the actual situation from the simulator together with the estimated situation from the GPR. The final results of this work shows acceptable accuracy and serve as a proof-of-concept that mobile phones can be used to map fire crisis situation. 
\end{abstract} 