\section{Fire Interpreter}

The fire interpreter's job is to receive input from the simulator and 
calculate values for the received cells. The calculated values are then posted to the visualizer. The advanced calculations are done by the library pyXGPR. This is a Gaussian Process Regression library implemented with Python. It produces a mean and a variance when used correctly. The first input parameter \textbf{X} is a list of points which tells where the training data is located. Another parameter \textbf{Y} contains the values to the training data. The last interpreter generated parameter \textbf{x star} contains the points where we want to find the mean and the variance. In addition to these parameters the library needs to be told what covariance functions pyXGPR should use to calculate the correlation between the cells in \textbf{X}, \textbf{Y} and \textbf{x star}. There is also added parameter values to these functions.
\\
The most basic use of pyXGPR is one dimensional (line regression) where \textbf{X} is the location and \textbf{Y} is the value. The interpreter uses \textbf{X} and \textbf{Y} for the map coordinate and an additional parameter \textbf{t} for time. \textbf{t} is necessary to save earlier sensor data which later are utilized in calculations. It should also be mentioned that before this implementation, this was done by saving the best data. Best data is to be understood as the data which has the lowest variance. Data with lower variance would be applied to the saved map. This hack and the implementation of t is done because previous sensor data is important as long as they are weighted less than the newest sensor data. As time increases there will be sensor data covering most of the map, but the old sensor data will have less weight and thus giving new sensor data the opportunity to be taken into account.