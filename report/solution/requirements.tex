\section{Requirements}
The requirements in this project could be derived from the very definition of crisis mapping: collect, interpret and visualize crisis data.

Data need to be collected by mobile phone sensors readings. It can be representative, meaning that it does not need to be real world values. Nevertheless, it should be possible to distinguish between high and low values and the physics still needs to be real. This way, it is not necessary to start real fires to obtain some real data. It is adequate to simulate the data in order to prove or disprove that the crisis mapping using mobile phone sensors work. However, there are conclusions that can not be drawn solely by individual sensor readings.

Sensor readings can and need to be interpreted on a collective basis in order to get more data out of the related crisis situation. The interpretation should be able to estimate the actual situation based on a limited number of sensors. This is due to the fact that there is no way to guarantee that there are adequate number of sensors available at the crisis location at any time. After the interpretation, the collected data and the data provided by the interpretation are still not humanly readable and therefore needs to be visualized.

The interpreted situation needs to be visualized so that crisis management teams could use it to make better decisions. It is adequate that the estimated values can be read and compared with the actual values to visually determine the performance of the system.